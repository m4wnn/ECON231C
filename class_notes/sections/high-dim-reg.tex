\section{High-Dimensional Linear Regression}

Consider the regression: 

\begin{align*}
    Y = \Vec{X}'\Vec{\beta} + \epsilon
\end{align*}

Where $Y$ is a scalar, $\Vec{X}$ is a $p$-dimensional vector of regressors, $\Vec{\beta}$ is a $p$-dimensional vector of coefficients, and $\epsilon$ is a scalar error term. And suppose we have a random sample:

\begin{align*}
    (\Vec{X}_1, Y_1), \cdots, (\Vec{X}_n, Y_n) \sim \text{i.i.d.} (\Vec{X}, Y)
\end{align*}

We are interested in case where $p$ is large, meaning that: $p \sim n$, $p > n$ or $p >> n$.

\begin{claim*}
    \emph{OLS} linear estimator does not exist when $p > n$.

    \begin{proof}
        The OLS linear regression estimator is given by:

        \begin{align*}
            \hat{\Vec{\beta}} = (\Mat{X}'\Mat{X})^{-1}\Mat{X}'\Vec{Y}
        \end{align*}

        Where $\Mat{X} = (\Vec{X_1}, \cdots, \Vec{X_n})'$ is the $n \times p$ design matrix, and $\Vec{Y} = (Y_1, \cdots, Y_n)'$ is the $n \times 1$ vector of dependent variables. 

        \begin{lemma*}
            If $\Mat{A}$ is degenerated, then $\Mat{A}^{-1}$ does not exist.

            \begin{def*}
            A matrix $\Mat{A}$ is degenerated if there exists a non-zero vector $\Vec{v}$ such that $\Mat{A}\Vec{v} = \Vec{0}$.
            \end{def*}

            \begin{proof}
                Asume that $\Mat{A}$ is invertible, then there exist $\Mat{A}\Mat{A}^{-1} = \Mat{I}_p$. If $\Mat{A}$ is degenerated, then there exists a non-zero vector $\Vec{v}$ such that $\Mat{A}\Vec{v} = \Vec{0}$. Multiplying both sides by $\Mat{A}^{-1}$ we get:

                \begin{align*}
                    \Mat{A}^{-1}\Mat{A}\Vec{v} = \Mat{A}^{-1}\Vec{0} \implies \Vec{v} = \Vec{0}
                \end{align*}

                Which is a contradiction. Therefore, $\Mat{A}$ is not invertible.
            \end{proof}
        \end{lemma*}

        \begin{lemma*}
            The matrix $\Mat{A} = \Mat{X}'\Mat{X}$ is degenerated when $p > n$.

            \begin{proof}
                Consider the linear system of equations:
                \begin{align*}
                \Mat{X}\Vec{b} = \Vec{0}, \quad \Vec{b} \in \mathbb{R}^p
                \end{align*}

                Where $\Mat{X}$ is a $n \times p$ matrix. This system has a non unique solution when $p > n$. therefore, there existe a non-zero vector $\Vec{b}$ such that $\Mat{X}\Vec{b} = \Vec{0}$, meaning that $\Mat{X}'\Mat{X}$ is degenerated.
            \end{proof}
        \end{lemma*}

        From the previous lemma, we know that $\Mat{X}'\Mat{X}$ is degenerated when $p > n$. Therefore, $\Mat{X}'\Mat{X}$ is not invertible, and the OLS linear estimator does not exist.
    \end{proof}
\end{claim*}

%%%%%%%%%%%%%%%%%%%%%%%%%%%%%%%%%%%%%%%%%%%%%%%%%%%%%%%%%%%%%%%%%%%%%%%%%%%
\subsection{Expected Value of the OLS Linear Regression Estimator when $p > n$}

Assuming that the linear model is homoscedastic, that is: 

\begin{align*}
    \E{\epsilon^2 | \Vec{X}} = \sigma^2
\end{align*}

then: 

\begin{align*}
\E{
    \frac{1}{n}
    \sum_{i=1}^{n} \left(\Vec{X}_i' 
        (\hat{\Vec{\beta}}^{OLS} - \Vec{\beta})
    \right)^2
}
=
\dfrac{p \sigma^2}{n}
\end{align*}

\begin{proof}

Given the linear regression in matrix form:

\begin{align*}
    \Vec{Y} = \Mat{X}\Vec{\beta} + \Vec{\epsilon}
\end{align*}

Where $\Vec{Y} = (Y_1, \cdots, Y_n)'$ and $\Vec{\epsilon} = (\epsilon_1, \cdots, \epsilon_n)'$ are $n \times 1$ column vectors; $\Vec{\beta} = (\beta_1, \cdots, \beta_p)'$ is a $p \times 1$ column vector; and $\Mat{X} = (\Vec{X}_1, \cdots, \Vec{X}_n)'$ is a $n \times p$ matrix.

Recall that the OLS linear estimator is given by:

\begin{align*}
    \hat{\Vec{\beta}}^{OLS} = (\Mat{X}'\Mat{X})^{-1}\Mat{X}'\Vec{Y}
\end{align*}

Substituting the linear regression in matrix form into the OLS linear estimator we get:

\begin{align*}
    \hat{\Vec{\beta}}^{OLS} &= (\Mat{X}'\Mat{X})^{-1}\Mat{X}'(\Mat{X}\Vec{\beta} + \Vec{\epsilon}) \\
    \hat{\Vec{\beta}}^{OLS} - \Vec{\beta} &= (\Mat{X}'\Mat{X})^{-1}\Mat{X}'\Vec{\epsilon} \\
    \Mat{X}(\hat{\Vec{\beta}}^{OLS} - \Vec{\beta}) &= \Mat{X}(\Mat{X}'\Mat{X})^{-1}\Mat{X}'\Vec{\epsilon} \\
\end{align*}

Where: 

\begin{align*}
    \Vec{e} = \widehat{\Vec{Y}} - \E{\Vec{Y} | X} = \Mat{X}(\hat{\Vec{\beta}}^{OLS} - \Vec{\beta}) 
\end{align*}

Therefore, the expected value of the mean squared error for the OLS estimator is:

\begin{align*}
\E{
    \frac{1}{n}
    \sum_{i=1}^{n} \left(\Vec{X}_i' 
        (\hat{\Vec{\beta}}^{OLS} - \Vec{\beta})
    \right)^2
}
&=
\frac{1}{n} \E{
    \Vec{e}'\Vec{e}
}\\
&= \frac{1}{n} \E{
    (\Mat{X}(\Mat{X}'\Mat{X})^{-1}\Mat{X}'\Vec{\epsilon})'(\Mat{X}(\Mat{X}'\Mat{X})^{-1}\Mat{X}'\Vec{\epsilon})
}\\
&= \frac{1}{n} \E{
    \Vec{\epsilon}'\Mat{X}(\Mat{X}'\Mat{X})^{-1}\Mat{X}'\Mat{X}(\Mat{X}'\Mat{X})^{-1}\Mat{X}'\Vec{\epsilon}
}\\
\end{align*}
\begin{align}
\E{
    \frac{1}{n}
    \sum_{i=1}^{n} \left(\Vec{X}_i' 
        (\hat{\Vec{\beta}}^{OLS} - \Vec{\beta})
    \right)^2
}
&= \frac{1}{n} \E{
    \Vec{\epsilon}'\Mat{X}(\Mat{X}'\Mat{X})^{-1}\Mat{X}'\Vec{\epsilon}
}
\label{eq:expected-value-mse}
\end{align}

\begin{lemma*}
    If $\Mat{A}$ is a $n \times p$ matrix, and $\Mat{B}$  is a $p \times n$ matrix, then:

    \begin{align*}
        \tr{\Mat{A}\Mat{B}} = \tr{\Mat{B}\Mat{A}}
    \end{align*}
\end{lemma*}

The expected value in equation \ref{eq:expected-value-mse} is an scalar, therefore:

\begin{align*}
\frac{1}{n} \E{
    \Vec{\epsilon}'\Mat{X}(\Mat{X}'\Mat{X})^{-1}\Mat{X}'\Vec{\epsilon}
}
&= \frac{1}{n} \tr{\E{
    \Vec{\epsilon}'\Mat{X}(\Mat{X}'\Mat{X})^{-1}\Mat{X}'\Vec{\epsilon}
}}
\end{align*}

Because the trace and the expected value are linear operators, we interchange the order of the trace and the expected value:

\begin{align*}
\frac{1}{n} \E{
    \Vec{\epsilon}'\Mat{X}(\Mat{X}'\Mat{X})^{-1}\Mat{X}'\Vec{\epsilon}
}
&= \frac{1}{n} \E{\tr{
    \Vec{\epsilon}'\Mat{X}(\Mat{X}'\Mat{X})^{-1}\Mat{X}'\Vec{\epsilon}
}} \\
&= \frac{1}{n} \E{\tr{
    \Vec{\epsilon}\Vec{\epsilon}'\Mat{X}(\Mat{X}'\Mat{X})^{-1}\Mat{X}'
}}\\
&= \frac{1}{n} \tr{\E{
    \Vec{\epsilon}\Vec{\epsilon}'\Mat{X}(\Mat{X}'\Mat{X})^{-1}\Mat{X}'
}}
\end{align*}

By the law of iterated expectations:

\begin{align*}
\frac{1}{n} \E{
    \Vec{\epsilon}'\Mat{X}(\Mat{X}'\Mat{X})^{-1}\Mat{X}'\Vec{\epsilon}
}
&= \frac{1}{n} \tr{\E{\E{
    \Vec{\epsilon}\Vec{\epsilon}'\Mat{X}(\Mat{X}'\Mat{X})^{-1}\Mat{X}'
 | \Mat{X}} }}\\
&= \frac{1}{n} \tr{\E{\E{
    \Vec{\epsilon}\Vec{\epsilon}'| \Mat{X}} \Mat{X}(\Mat{X}'\Mat{X})^{-1}\Mat{X}'}}\\
&= \frac{1}{n} \tr{
        \E{
            \sigma^2 \Mat{I}_n
            \Mat{X}(\Mat{X}'\Mat{X})^{-1}\Mat{X}'
        }
    }\\
&= \frac{\sigma^2}{n} \tr{
        \E{
            \Mat{X}(\Mat{X}'\Mat{X})^{-1}\Mat{X}'
        }
    }\\
&= \frac{\sigma^2}{n} \E{
        \tr{
            \Mat{X}(\Mat{X}'\Mat{X})^{-1}\Mat{X}'
        }
    }\\
&= \frac{\sigma^2}{n} \E{
        \tr{
            \Mat{X}'\Mat{X}(\Mat{X}'\Mat{X})^{-1}
        }
    }\\
&= \frac{\sigma^2}{n} \E{
        \tr{
           \Mat{I}_p 
        }
    }\\
&= \frac{\sigma^2}{n} \E{p}\\
\end{align*}

\begin{align*}
\frac{1}{n} \E{
    \Vec{\epsilon}'\Mat{X}(\Mat{X}'\Mat{X})^{-1}\Mat{X}'\Vec{\epsilon}
}
&= \frac{p\sigma^2}{n}\\
\end{align*}

\end{proof}

Meaning that the expected value of the mean squared error of the OLS linear estimator does not converge to zero when $p >> n$.