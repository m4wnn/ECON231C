\documentclass[letterpaper]{article}

% No page numbers.
\pagestyle{empty}

% Change the margins to 1 inch all around.
\usepackage[margin=1in]{geometry}

% for setting the linespace (\setstretch)
\usepackage{setspace}

% distance between the columns (for the multicols)
\setlength{\columnsep}{1cm}

% for compactitem
\usepackage{paralist}

% Hyperlinks inside the document.
\usepackage{hyperref}

\usepackage{verbatim}
\usepackage{xcolor}
\usepackage{booktabs}
\usepackage{longtable}
\usepackage{float}
\usepackage{graphicx}
\usepackage{listings}
\usepackage{amsmath}
\usepackage{tcolorbox}
\usepackage{amssymb}
\usepackage{parskip}
\usepackage{breqn}
\usepackage{booktabs}
\usepackage{pdfpages}
\usepackage{multirow}
\usepackage{parskip}
\usepackage{array}
\usepackage{amsthm} 

% for the bibliography
\usepackage[authordate, backend=biber, natbib=true]{biblatex-chicago}
\addbibresource{bibliography.bib} % Link to your bibliography file

% lorem ipsum generator.
\usepackage{lipsum}

%%%%% Extra math symbols %%%%%
\DeclareMathOperator*{\argmin}{arg\,min}
\DeclareMathOperator*{\argmax}{arg\,max}
\DeclareMathOperator*{\plim}{plim}

%%%%% Extra Commands %%%%%
\newcommand{\E}[1]{\mathbb{E}\left[#1\right]}
\newcommand{\indicator}[1]{\mathbf{1}_{\{#1\}}}
\newcommand{\prob}[1]{\mathbb{P}\left(#1\right)}
\newcommand{\var}[1]{\text{Var}\left(#1\right)}
\newcommand{\cov}[1]{\text{Cov}\left(#1\right)}
\newcommand{\corr}[1]{\text{Corr}\left(#1\right)}
\newcommand{\convP}{\xrightarrow{\mathbb{P}}}
\newcommand{\Exp}[1]{\exp\left(#1\right)}
\renewcommand{\Vec}[1]{\mathbf{#1}}
\newcommand{\Mat}[1]{\mathbb{#1}}
\newcommand{\tr}[1]{\text{tr}\left(#1\right)}


%%%%% Color definitions %%%%%
\definecolor{lightgray}{gray}{-1.9}
\definecolor{codegreen}{rgb}{0,0.6,0}
\definecolor{codegray}{rgb}{0.5,0.5,0.5}
\definecolor{codepurple}{rgb}{0.58,0,0.82}

%%%%%%%%%%%%%%%%%%%%%%%%%%%%%%%%%%%%%%%%%%%%%%%%%%%%%%%%%%%%%%%%%
\begin{document}

%%%%% Extra Environments %%%%%
\newtheorem*{theorem*}{Theorem}
\newtheorem*{claim*}{Claim}
\newtheorem*{lemma*}{Lemma}
\newtheorem*{def*}{Definition}

%%%%% Extra Environments %%%%%
\newtheorem{theorem}{Theorem}
\newtheorem{claim}{Claim}
\newtheorem{lemma}{Lemma}
\newtheorem{Def}{Definition}

%%%%% Answer box environment %%%%%
\newtcolorbox{myanswerbox}[1][]{
  colback=white!, % Background color
  colframe=white!25!black, % Bourder color
  fonttitle=\bfseries, %  Font style of the title
  title=#1, %  Title
  arc=0mm, %  Rounded corners
  #1
}

% Style configuration for the code listings.
\lstdefinestyle{mystyle}{
    commentstyle=\color{codegreen},
    keywordstyle=\color{magenta},
    numberstyle=\tiny\color{codegray},
    stringstyle=\color{codepurple},
    basicstyle=\ttfamily\footnotesize,
    breakatwhitespace=false,         
    breaklines=true,                 
    captionpos=b,                    
    keepspaces=true,                 
    numbers=left,                    
    numbersep=5pt,                  
    showspaces=false,                
    showstringspaces=false,
    showtabs=false,                  
    tabsize=2
}

\lstset{style=mystyle}

%=============================
{
	\parindent0pt
	\setstretch{0.4}
	\ \\ \ \\ \ \\

	\hrulefill
	\vspace{0.0cm}
	\begin{spacing}{1.1}
	{	
		\flushleft
		\fontsize{22pt}{44pt}\selectfont 
		Class Notes
	}\\
	\textsc{Estimation in High-Dimensional Spaces - ECON231C}
	\end{spacing}

	\ \\ \ \\
	{
		\setstretch{0.2}
		\textbf{Mauricio Vargas-Estrada}\\
		Master in Quantitative Economics\\
		University of California - Los Angeles\par
	}
	\ \\

	\hrulefill
}

%\newpage
%=============================
\section*{Problem 1}
%%%%% PROBLEM STATEMENT %%%%%%%%%%%%%%%%%%%%%%%%%%%%%%%%%%%%%%%%%
\begin{myanswerbox}
Prove Hoeffding's lemma: for any mean-zero random variable \(Z\) satisfying \(|Z| \leq a\) almost surely,
\[ 
    \E{\Exp{\lambda Z_i}} \leq \exp\left(\frac{\lambda^2 a^2}{2}\right), 
\]
for all \(\lambda > 0\).
\end{myanswerbox}
%%%%% QUESTION SEPARATOR %%%%%%%%%%%%%%%%%%%%%%%%%%%%%%%%%%%%%%%%
%%%%% Answer %%%%%
\begin{proof}
Since $\Exp{\lambda Z_i}$ is a convex function of $Z_i$, for all $z_i \in [-a, a]$,

\begin{align*}
\Exp{\lambda z_i}
&\leq 
\dfrac{a - z_i}{a - (-a)} \Exp{ -a \lambda} + 
\dfrac{z_i - (-a)}{a - (-a)} \Exp{a \lambda} \\
%%%
&\leq 
\dfrac{a - z_i}{2a} \Exp{ -a \lambda} + 
\dfrac{z_i + a}{2a} \Exp{a \lambda} \\
\end{align*}

taking expectation on both sides and knowing that $Z$ is mean-zero, we have,


\begin{align*}
\E{\Exp{\lambda Z_i}}
&\leq 
\E{\dfrac{a - Z_i}{2a} \Exp{ -a \lambda}} + 
\E{\dfrac{Z_i + a}{2a} \Exp{a \lambda}} \\
%%%
&\leq 
\dfrac{a - \E{Z_i}}{2a} \Exp{ -a \lambda} + 
\dfrac{\E{Z_i} + a}{2a} \Exp{a \lambda} \\
%%%
&\leq 
\dfrac{a}{2a} \Exp{ -a \lambda} + 
\dfrac{a}{2a} \Exp{a \lambda} \\
%%%
&\leq 
\dfrac{1}{2} \Exp{ -a \lambda} + 
\dfrac{1}{2} \Exp{a \lambda} \\
&\leq 
\Exp{ -a \lambda} + 
\Exp{a \lambda} \\
\end{align*}

Calculating the Taylor's expansion of the right-hand side of the inequality over $\lambda$ and simplifying,

\begin{align*}
\Exp{ -a \lambda} +
\Exp{a \lambda}
&=
\left(1 - a \lambda + \dfrac{a^2 \lambda^2}{2} - \dfrac{a^3 \lambda^3}{6} + \ldots\right) +
\left(1 + a \lambda + \dfrac{a^2 \lambda^2}{2} + \dfrac{a^3 \lambda^3}{6} + \ldots\right) \\
%%%
&=
1 + \dfrac{a^2 \lambda^2}{2} + \dfrac{a^4 \lambda^4}{24}\ldots\\
\end{align*}

We recognize that the Taylor's expansion of $\Exp{\dfrac{a^2\lambda^2}{2}}$ is:

\begin{align*}
\Exp{\dfrac{a^2\lambda^2}{2}} = 1 + \dfrac{a^2 \lambda^2}{2} + \dfrac{a^4 \lambda^4}{8}\ldots
\end{align*}

meaning that:

\begin{align*}
1 + \dfrac{a^2 \lambda^2}{2} + \dfrac{a^4 \lambda^4}{24}\ldots
& \leq
1 + \dfrac{a^2 \lambda^2}{2} + \dfrac{a^4 \lambda^4}{8}\ldots\\
%%%
\Exp{ -a \lambda} + \Exp{a \lambda} &\leq \Exp{\dfrac{a^2\lambda^2}{2}}\\ 
\end{align*}

Therefore

\begin{align*}
\E{\Exp{\lambda Z_i}} \leq \exp\left(\frac{\lambda^2 a^2}{2}\right)
\end{align*}
\end{proof}
\section*{Problem 2}
%%%%% PROBLEM STATEMENT %%%%%%%%%%%%%%%%%%%%%%%%%%%%%%%%%%%%%%%%%
\begin{myanswerbox}
While proving the Hoeffding inequality, we said that two probabilities,
\[
    \prob{\frac{1}{n} \sum_{i=1}^{n} X_i - \mu \geq t }
\]
and
\[ 
    \prob{\frac{1}{n} \sum_{i=1}^{n} X_i - \mu \leq -t },
\]
can be bounded in the same way and did the derivation only for the former probability. Show that the latter probability is indeed bounded by the same quantity.
\end{myanswerbox}
%%%%% QUESTION SEPARATOR %%%%%%%%%%%%%%%%%%%%%%%%%%%%%%%%%%%%%%%%
%%%%% Answer %%%%%

\begin{proof}
    Lets define $Z_i = X_i - \mu, \forall i = 1, 2, \dots, n$. Then, we have:
    
    \begin{align*}
    \bar{Z}_n = \frac{1}{n} \sum_{i=1}^n Z_i = \bar{X}_n - \mu
    \end{align*}
    
    and,
    
    \begin{align*}
    |Z_i| \leq a, \forall i = 1, 2, \dots , n
    \end{align*}
    
    then, whe can rewrite the former probability as:
    
    \begin{align*}
    \prob{\frac{1}{n} \sum_{i=1}^{n} X_i - \mu \leq -t } = \prob{\bar{Z}_n \geq -t}
    \end{align*}
    
    Expanding the average $\bar{Z}_n$:
    \begin{align*}
    \prob{\bar{Z}_n \geq -t} = \prob{\sum_{i=1}^n Z_i \leq -nt}
    \end{align*}
    
    for any $\lambda > 0$, we have \footnote{Given that $f(x) = \lambda x$ is a monotonically increasing function when $\lambda > 0$}:
    
    \begin{align*}
    \prob{\bar{Z}_n \geq -t} = 
    \prob{\lambda \sum_{i=1}^n Z_i \leq -\lambda nt}\\
    \end{align*}
    
    and, by Markov's inequality:
    
    \begin{align*}
    \prob{\lambda \sum_{i=1}^n Z_i \leq -\lambda nt} 
    \leq
    \frac{\E{\exp(\lambda \sum_{i=1}^n Z_i)}}{\exp(-\lambda nt)}
    \end{align*}
    
    since $Z_i$ are independent and identically distributed, we can write:
    
    \begin{align*}
    \frac{\E{\exp(\lambda \sum_{i=1}^n Z_i)}}{\exp(-\lambda nt)} 
    &=
    \frac{\prod_{i=1}^n \E{\exp(-\lambda Z_i)}}{\exp(-\lambda nt)}\\
    %%%%%
    &=
    \frac{\prod_{i=1}^n \exp(-1)\E{\exp(\lambda Z_i)}}{\exp(-\lambda nt)}\\
    %%%%%
    &=
        \frac{\prod_{i=1}^n \E{\exp(\lambda Z_i)}}{\exp(\lambda nt)} 
    \end{align*}
    
    Applying the Hoeffding's lemma \footnote{
        If $X$ is a random variable such that $X \leq a$, then for any $\lambda > 0$, we have:
        \begin{align*}
        \E{\exp(\lambda X)} 
        \leq
        \exp\left(\frac{\lambda^2a^2}{2}\right)
        \end{align*}
    } to the above expression, we have:
    
    \begin{align*}
    \frac{\prod_{i=1}^n \E{\exp(\lambda Z_i)}}{\exp(\lambda nt)} 
    &\leq
    \frac{\prod_{i = 1}^n \exp\left(\frac{\lambda^2a^2}{2}\right)}{\exp(\lambda nt)}\\
    %%%%%
    &\leq
    \frac{\exp\left(\frac{n\lambda^2a^2}{2}\right)}{\exp(\lambda nt)}\\
    %%%%%
   &\leq
    \exp\left(\frac{n\lambda^2a^2}{2} - \lambda nt \right)\\
    \end{align*}
    
    Because the above inequality holds for any $\lambda > 0$, we can optimize the right-hand side with respect to $\lambda$.
    
    \begin{align*}
    \lambda^* = \argmin_{\lambda > 0} \left\{ \frac{n\lambda^2a^2}{2} - \lambda nt \right\}
    \end{align*}
    
    Calculating the F.O.C. with respect to $\lambda$, we get:
    
    \begin{align*}
    n a^2 \lambda^* - nt = 0 \Rightarrow \lambda^* = \frac{t}{a^2}
    \end{align*}
    
    Substituting $\lambda^*$ back into the inequality:
    
    \begin{align*}
    \prob{\bar{Z}_n \geq -t} &\leq
        \exp\left(\frac{n\left(\frac{t}{a^2}\right)^2a^2}{2} - \frac{t}{a^2} n t \right)\\
    %%%%%
    &\leq
        \exp\left(\frac{nt^2}{2a^2} - \frac{nt^2}{a^2} \right)\\
    %%%%%
    &\leq
        \exp\left(-\frac{nt^2}{2a^2} \right)\\
    \end{align*}
    
    replacing $\bar{Z}_n$ by $\bar{X}_n - \mu$:
    
    \begin{align*}
    \prob{|\bar{X}_n - \mu| \geq t} \leq \exp\left(-\frac{nt^2}{2 a^2}\right)
    \end{align*}
    
    \end{proof}
\section*{Problem 3}
%%%%% PROBLEM STATEMENT %%%%%%%%%%%%%%%%%%%%%%%%%%%%%%%%%%%%%%%%%
\begin{myanswerbox}
(Tricky) While proving the Hoeffding inequality, we have used the following bound:
\[ P(X > \delta) \leq \frac{E[e^{\lambda X}]}{e^{\lambda \delta}}, \quad \lambda > 0. \]
An alternative could be
\[ P(X > \delta) \leq \frac{E[|X|^k]}{\delta^k}, \quad k \geq 0. \]
Show that if \( X \geq 0 \) a.s., then
\[ \inf_{k=0,1,2,\ldots} \frac{E[|X|^k]}{\delta^k} \leq \inf_{\lambda>0} \frac{E[e^{\lambda X}]}{e^{\lambda \delta}}. \]
\end{myanswerbox}
%%%%% QUESTION SEPARATOR %%%%%%%%%%%%%%%%%%%%%%%%%%%%%%%%%%%%%%%%
%%%%% Answer %%%%%
\section*{Problem 4}
%%%%% PROBLEM STATEMENT %%%%%%%%%%%%%%%%%%%%%%%%%%%%%%%%%%%%%%%%%
\begin{myanswerbox}
While proving the maximal inequality, i.e., a bound on
\[ P\left(\max_{1 \leq j \leq p} \left| \frac{1}{n} \sum_{i=1}^{n} X_{ij} - \mu_j \right| \geq t \right), \]
we applied the union bound followed by the Hoeffding inequality. Show what happens if we replace the Hoeffding inequality by the Chebyshev inequality.
\end{myanswerbox}
%%%%% QUESTION SEPARATOR %%%%%%%%%%%%%%%%%%%%%%%%%%%%%%%%%%%%%%%%
%%%%% Answer %%%%%

%\newpage
%=============================
%\section*{References}
% the \nocite command leads to the whole bibliography
% being displayed (without any \cite commands necessary).
% remove this command in order to get the "normal" behavior.
%\nocite{*}
%\newpage
%\printbibliography[heading=none]

\end{document}
