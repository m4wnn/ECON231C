\documentclass[letterpaper]{article}

% No page numbers.
\pagestyle{empty}

% Change the margins to 1 inch all around.
\usepackage[margin=1in]{geometry}

% for setting the linespace (\setstretch)
\usepackage{setspace}

% distance between the columns (for the multicols)
\setlength{\columnsep}{1cm}

% for compactitem
\usepackage{paralist}

% Hyperlinks inside the document.
\usepackage{hyperref}

\usepackage{verbatim}
\usepackage{xcolor}
\usepackage{booktabs}
\usepackage{longtable}
\usepackage{float}
\usepackage{graphicx}
\usepackage{listings}
\usepackage{amsmath}
\usepackage{tcolorbox}
\usepackage{amssymb}
\usepackage{parskip}
\usepackage{breqn}
\usepackage{booktabs}
\usepackage{pdfpages}
\usepackage{multirow}
\usepackage{parskip}
\usepackage{array}
\usepackage{amsthm} 

% Bibliography management.
\usepackage[
  sorting=none,
  minbibnames=8,
  maxbibnames=9,
  block=space,
  backend=biber
]{biblatex}
\bibliography{bibliography}

% lorem ipsum generator.
\usepackage{lipsum}

%%%%% Extra math symbols %%%%%
\DeclareMathOperator*{\argmin}{arg\,min}
\DeclareMathOperator*{\argmax}{arg\,max}
\DeclareMathOperator*{\plim}{plim}

%%%%% Extra Commands %%%%%
\newcommand{\E}[1]{\mathbb{E}\left[#1\right]}
\newcommand{\indicator}[1]{\mathbf{1}_{\{#1\}}}
\newcommand{\prob}[1]{\mathbb{P}\left(#1\right)}
\newcommand{\var}[1]{\text{Var}\left(#1\right)}
\newcommand{\cov}[1]{\text{Cov}\left(#1\right)}
\newcommand{\corr}[1]{\text{Corr}\left(#1\right)}



%%%%% Color definitions %%%%%
\definecolor{lightgray}{gray}{-1.9}

%%%%%%%%%%%%%%%%%%%%%%%%%%%%%%%%%%%%%%%%%%%%%%%%%%%%%%%%%%%%%%%%%
\begin{document}

%%%%% Answer box environment %%%%%
\newtcolorbox{myanswerbox}[1][]{
  colback=white!, % Background color
  colframe=white!25!black, % Bourder color
  fonttitle=\bfseries, %  Font style of the title
  title=#1, %  Title
  arc=0mm, %  Rounded corners
  #1
}

%%%%% Python code environment %%%%%
\lstset{
    language=Python,
    basicstyle=\ttfamily\small,
    keywordstyle=\color{blue},
    backgroundcolor=\color{lightgray}
}

%=============================
{
	\parindent0pt
	\setstretch{0.4}
	\ \\ \ \\ \ \\

	\hrulefill
	\vspace{0.0cm}
	\begin{spacing}{1.1}
	{	
		\flushleft
		\fontsize{22pt}{44pt}\selectfont 
		Class Notes
	}\\
	\textsc{Estimation in High-Dimensional Spaces - ECON231C}
	\end{spacing}

	\ \\ \ \\
	{
		\setstretch{0.2}
		\textbf{Mauricio Vargas-Estrada}\\
		Master in Quantitative Economics\\
		University of California - Los Angeles\par
	}
	\ \\

	\hrulefill
}

%\newpage
%=============================
\section{Markov Inequality}

Being $X$ a random variable such that $X \geq 0$, then:

\begin{equation*}
\prob{X \geq t} \leq \frac{\E{X}}{t}, \quad \forall t > 0
\end{equation*}

\begin{proof}
Whe can rewrite the left-hand side of the inequality using the indicator function:

\begin{align*}
X \geq X \indicator{X \geq t}
\end{align*}

The left-hand side would be greater when $X < t$ and equal when $X \geq t$. Given that:

\begin{align*}
X \geq X \indicator{X \geq t} \geq t \indicator{X \geq t}
\end{align*}

In this case, $X \indicator{X \geq t} > t \indicator{X \geq t}$ when $X > t$, and $X \indicator{X \geq t} = t \indicator{X \geq t}$ when $X \leq t$ because $X = t$ or the indicator function is zero.

Taking the expectation of the inequality:

\begin{align*}
\E{X} &\geq \E{X \indicator{X \geq t}} \geq \E{t \indicator{X \geq t}} \\
\E{X} &\geq \E{X \indicator{X \geq t}} \geq t \E{\indicator{X \geq t}} \\
\frac{\E{X}}{t} &\geq \frac{\E{X \indicator{X \geq t}}}{t} \geq \E{\indicator{X \geq t}} \\
\end{align*}

But $\E{\indicator{X \geq t}} = \prob{X \geq t}$, so:

\begin{align*}
\frac{\E{X}}{t} &\geq \prob{X \geq t} \\
\end{align*}

\end{proof}
\section{Chevyshev Inequality}

Given a random variable $X$ with mean $\mu$ and variance $\sigma^2$, then:

\begin{align*}
\prob{|X - \mu| \geq t} \leq \frac{\sigma^2}{t^2}, \quad \forall t > 0
\end{align*}

\begin{proof}
We are going to use the fact that a strictly increasing function of a random variable does not change the probability of an event. Let $Y = (|X - \mu|)^2$. Then,

\begin{align*}
\prob{|X - \mu|^2 \geq t^2} &= \prob{Y \geq t^2} \\
\end{align*}

Using Markov's inequality, we have:

\begin{align*}
\prob{Y \geq t^2} &\leq \frac{\E{Y}}{t^2} \\
\end{align*}

Given a random variable $Z$, the variance of $Z$ is $\var{Z} = \E{(Z - \E{Z})^2}$. Therefore,

\begin{align*}
\prob{(X - \mu)^2 \geq t^2} &\leq \frac{\E{(X - \mu)^2}}{t^2} \\
\prob{(X - \mu)^2 \geq t^2} &\leq \frac{\var{X}}{t^2} \\
\prob{(X - \mu)^2 \geq t^2} &\leq \frac{\sigma^2}{t^2} \\
\end{align*}

\end{proof}


\end{document}