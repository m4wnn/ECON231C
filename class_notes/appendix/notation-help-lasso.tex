\section{Helpful Notation for LASSO Theory}
\label{sec:notation-help-lasso}

Recall the linear model:

\begin{align*}
Y = \Vec{X}'\Vec{\beta} + \epsilon, 
    \quad \E{\epsilon|\Vec{X}} = 0
\end{align*}
 
Where:
\begin{itemize}
\item Vector of $p$ covariates: $\Vec{X} = (X_1, \cdots, X_j, \cdots , X_p)' \in \mathbb{R}^p$
\item Dependent Variable: $Y \in \mathbb{R}$
\item Vector of coefficients: $\Vec{\beta} = (\beta_1, \cdots, \beta_j, \cdots, \beta_p)' \in \mathbb{R}^p$
\item Error term: $\epsilon \in \mathbb{R}$
\end{itemize}

and a random sample $(\Vec{X}_1, Y_1), \cdots, (\Vec{X}_n, Y_n) \overset{iid}{\sim} (\Vec{X}, Y)$, where $\Vec{X}_i \in \mathbb{R}^p$ and $Y_i \in \mathbb{R}$.

The linear regression (incorporating the random sample) is denoted as:

\begin{align*}
Y_i = \Vec{X}_i'\Vec{\beta} + \epsilon_i, 
    \quad \E{\epsilon_i|\Vec{X}_i} = 0, \quad \forall i = 1, \cdots, n
\end{align*}

We can express the same regression in matrix form:

\begin{align*}
\Vec{Y} = \Mat{X}'\Vec{\beta} + \Vec{\epsilon}, 
    \quad \E{\Vec{\epsilon}|\Mat{X}} = 0
\end{align*}

Where:
\begin{itemize}
\item Dependent Vector: $\Vec{Y} = (Y_1, \cdots, Y_i, \cdots, Y_n)' \in \mathbb{R}^n$
\item Design Matrix: 
    \begin{align*}
        \Mat{X} &= \left[
            \begin{array}{cccccc}
                X_{11} & \cdots & X_{1j} & \cdots & X_{1p} \\
                \vdots & \ddots & \vdots & \ddots & \vdots \\
                X_{i1} & \cdots & X_{ij} & \cdots & X_{ip} \\
                \vdots & \ddots & \vdots & \ddots & \vdots \\
                X_{n1} & \cdots & X_{nj} & \cdots & X_{np} \\
            \end{array}
        \right] \in \mathbb{R}^{n \times p}\\
        \Mat{X} &= \left[
            \begin{array}{cccccc}
                \Vec{X}_{1}' \\
                \vdots \\
                \Vec{X}_{i}' \\
                \vdots \\
                \Vec{X}_{n}' \\
            \end{array}
        \right] = (\Vec{X}_1, \cdots, \Vec{X}_i, \cdots, \Vec{X}_n)' \in \mathbb{R}^{n \times p}\\
        \Mat{X} &= (\Vec{X}_{(1)}, \cdots, \Vec{X}_{(j)}, \cdots, \Vec{X}_{(p)}) \in \mathbb{R}^{n \times p}\\
    \end{align*}
\item Vector of coefficients: $\Vec{\beta} = (\beta_1, \cdots, \beta_j, \cdots, \beta_p)' \in \mathbb{R}^p$
\item Error Vector: $\Vec{\epsilon} = (\epsilon_1, \cdots, \epsilon_i, \cdots, \epsilon_n)' \in \mathbb{R}^n$
\end{itemize}

Potentially, $p \sim n$, $p > n$ or $p >> n$.


The estimated linear regression is:
\begin{align*}
\widehat{\Vec{Y}} = \Mat{X}'\hat{\Vec{\beta}}
\end{align*}

Where, for \emph{OLS}, the vector estimator of coefficients is given by:

\begin{align*}
\hat{\Vec{\beta}} = (\Mat{X}'\Mat{X})^{-1}\Mat{X}'\Vec{Y}
\end{align*}

The error of estimation is given by:

\begin{align*}
\Vec{e} &= \E{\Vec{Y}|\Mat{X}} - \widehat{\Vec{Y}}\\
&= \Mat{X}'\Vec{\beta} - \Mat{X}'\hat{\Vec{\beta}}\\
&= \Mat{X}'(\Vec{\beta} - \hat{\Vec{\beta}}) \\
&= \Mat{X}'\Vec{\delta}
\end{align*}

which is often used to calculate the prediction norm:

\begin{align*}
\|\Vec{\beta} - \hat{\Vec{\beta}}\|_{2,n} 
    = \|\Vec{\delta}\|_{2,n} 
    = \sqrt{\frac{1}{n} \Vec{\delta}'\Mat{X}\Mat{X}'\Vec{\delta}} 
    = \sqrt{\frac{1}{n} \sum_{i=1}^{n} (\Vec{X}_i'\Vec{\delta})^2}
    = \sqrt{\frac{1}{n} \Vec{e}'\Vec{e}}
    = \rmse{\Mat{X}\hat{\Vec{\beta}}}
\end{align*}

\begin{Def}
Given a vector of  coefficients $\Vec{\beta} \in \mathbb{R}^p$, the sets $T$ and $T^{\complement}$ are defined as:

\begin{align*}
T &= \lbrace
j \in \{1, \cdots, p \} : \Vec{\beta}_j \neq 0, \forall 
\rbrace  \\
T^{\complement} &= \{1, 2, \cdots, p\} \setminus T
\end{align*}

where $T$ denotes the set of indices of the non-zero coefficients in $\Vec{\beta}$, and $T^{\complement}$ is the set of indices of the zero coefficients.

the cardinality of $T$ is $S = |T|$ and the cardinality of $T^{\complement}$ is $|T^{\complement}| = p - S$, where $S$ is the \emph{Sparsity Index}. By definition, $|T \cup T^{\complement}| = p$.
\end{Def}

\begin{Def}
Being $T$ and $T^{\complement}$ the sets of indices of the non-zero and zero coefficients in $\Vec{\beta}$, respectively, and a vector $\Vec{v} \in \mathbb{R}^p$, 

\begin{align*}
    \Vec{v}_T &= \begin{cases}
        v_j, &  j \in T \\
        0, &  j \notin T
    \end{cases}\\
    %%%
    \Vec{v}_{T^{\complement}} &= \begin{cases}
        v_j, &  j \in T^{\complement}\\
        0, &  j \notin T^{\complement}
    \end{cases}\\
\end{align*}

Are the projections of $\Vec{v}$ onto the sets $T$ and $T^{\complement}$, respectively.

By definition:

\[
\Vec{v} = \Vec{v}_T + \Vec{v}_{T^{\complement}}
\]
\end{Def}

\begin{Def}[Compatibility Constant]
$\forall c > 0$,

\begin{align*}
k_c &= \inf_{\Vec{\delta} \in \mathbb{R}^p}{
    \dfrac{\sqrt{S} \|\Vec{\delta}\|_{2, n}}{\|\Vec{\delta}\|_1}
}\\
&\text{s.t.}\\
&\|\Vec{\delta}_{T^{\complement}}\|_1 \leq c \|\Vec{\delta}_T\|_1
\end{align*}

where $\Vec{\delta} = \hat{\Vec{\beta}} - \Vec{\beta}$ is the diference between the estimated and true coefficients, and $\|\Vec{\delta}\|_{2, n} = \sqrt{\frac{1}{n} \Vec{\delta}'\Mat{X}\Mat{X}\Vec{\delta}}$ where $\Mat{X} = (\Vec{X}_1, \cdots, \Vec{X}_n)'$ is the design matrix.

The restriction set is often denoted as:
$\mathcal{R}_{c} = \{\Vec{\delta} \in \mathbb{R}^p: \|\Vec{\delta}_{T^\complement}\| \leq c \|\Vec{\delta}_T\|_1\|\}$
\end{Def}


\begin{Def}[Compatibility Constant (Alternative)]
$\forall \bar{c} > 0$,

\begin{align*}
k_{\bar{c}} &= \inf_{\Vec{\delta} \in \mathbb{R}^p}{
    \dfrac{\sqrt{S} \|\Vec{\delta}\|_{2, n}}{\|\Vec{\delta}\|_1}
}\\
&\text{s.t.}\\
&\|\Vec{\delta}_{T^{\complement}}\|_1 \leq \bar{c} \|\Vec{\delta}_T\|_1
\end{align*}

where $\Vec{\delta} = \hat{\Vec{\beta}} - \Vec{\beta}$ is the diference between the estimated and true coefficients, and $\|\Vec{\delta}\|_{2, n} = \sqrt{\frac{1}{n} \Vec{\delta}'\Mat{X}\Mat{X}\Vec{\delta}}$ where $\Mat{X} = (\Vec{X}_1, \cdots, \Vec{X}_n)'$ is the design matrix.

The restriction set is often denoted as $\mathcal{R}_{\bar{c}} = \{\Vec{\delta} \in \mathbb{R}^p: \|\Vec{\delta}_{T^\complement}\| \leq \bar{c} \|\Vec{\delta}_T\|_1\|\}$, and $\bar{c} = \dfrac{c+1}{c-1}$.
\end{Def}
